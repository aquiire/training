\section{vcLABs (Pvt.s) Ltd.}

\subsection{Information about the Training Establishment}

\begin{itemize}
\item Name: vcLABs (AI Research Centre) (Visual Concept Labs)
\item Address (Colombo Chapter): 3\textsuperscript{rd} Floor, No. 20, Arthur's Court, Vauxhall Street, Colombo (Fig. \ref{fig:vc-map})
\item Chief Executive Officer: Dr. Madhawa Silva (BSc. (UoM), MSc. (USA), PhD (USA), MIEEE)
\item Chief Architect: Eng. Sagara Wijeweera
\item Senior Technical Officer: Eng. Dilaj Perera 
\item Branches: Sri Lanka, USA, Australia
\end{itemize}
\begin{figure}[!hbt]
		\begin{center}
		\includegraphics [width=.4\textwidth]{vc-map.png}
		\caption{vcLABs Location, Arthurs Court, opposite to the Sri Lanka Insurance building in Vauxhall Street, Slave Island. }.
		\label{fig:vc-map}
		\end{center}
\end{figure}
\subsection{History}
vcLABs was founded by two graduates of University of Moratuwa, Pubudu Madhawa and Sagara Wijeweera on April 2, 2015. Madhawa followed Electronics and Telecommunications Engineering while Sagara graduated in Electrical Engineering. But they both were interested in computer science.

Around 2015, Sri Lankan government put forward a new legislation to protect the Sri Lankan producers. Airing foreign advertisements without proper monetization in Sri Lanka was forbidden under that legislation . So, DialogTV which is a popular satellite TV provider in Sri Lanka had to take immediate action to remove foreign ads from the channels they provide. Most of the time, the advertisements were Indian ads and it was reported that the products that was "freely" advertised were exported to Sri Lanka, directly exploiting the existing system. So DialogTV approached the scenario with a manual ad replacement system which replaced foreign ads with local ads which in turn created a extra revenue to DialogTV. The manual system was not due to some reasons. Some main reasons are listed below:

\begin{itemize}
\item Cost of the person who replace the ads
\item Cost for equipment which are specially required for manual system
\end{itemize}

So meantime, DialogTV researched for an automated system that performs the current ad replacement. They were not able to find such system. This is where Madhawa and Sagara comes up with a state of the art automated ad replacement system which even exceeded DialogTV's expectations. So this ad replacement system can be considered as world's first commercially used such a system.

Eventually Madhawa and Sagara decided to start up their own company in order to pioneer AI research in Sri Lanka. They named it Visual Concept Labs (or vcLABs).
 

\subsection{Company Logo}
Figure \ref{fig:logo} is the logo of vcLABs. It primarily depicts the idea "visual concepts". We perceive color and geometry as humans. It is the primary motivation behind the logo.
\begin{figure}[!hbt]
		\begin{center}
		\includegraphics [width=.4\textwidth]{vclogo.png}
		\caption{vcLABs Logo }.
		\label{fig:logo}
		\end{center}
\end{figure}

\section{vcLABs; Functions and Products}
\subsection{Colombo Chapter}
The core product in Colombo chapter is the ad replacement system which is functional in DialogTV. Currently the sole client in Colombo chapter is DialogTV. Dr. Madhawa and Eng. Sagara oversees the performance of the chapter from USA and Australia. The core product, the ad replacement system comprises of three parts:

\begin{itemize}
\item Advertisement Detection Tool: This tool detects foreign ads. Currently this system can handle a database of more than 100000 ads.
\item Advertisement Monitoring tool: This tool monitors the advertisement detection tool
\item Advertisement Insertion tool: This tool is responsible of insertion of local ads to the up stream. This tool performs dynamically minimizing removal of programs with buffering capability.
\end{itemize}

Other than the core product, the Colombo chapter is currently in the process of developing a novel system that would revolutionize ad market that comprises of retrieving and scheduling ads directly, without the intervention of an advertising agency as such.

Colombo Chapter's main function is maintaining its core product while developing the aforementioned new product. They are in constant touch with DialogTV in relation to those aspects.

\subsection{USA Branch}
USA branch which is solely led by Dr. Madhawa operates in a novel business model. It is mostly research led and has a researcher-network and make collaborations possible when they are available. The relationship between a researcher and the company is of four kinds:

\begin{itemize}
\item IP: The company shares the IP rights with the researcher(s)
\item Patents: The company shares the patents with the researcher(s)
\item Public: The research is published to the general public
\item Trade Secret: The research is kept as a trade secret by the company. A royalty is negotiated between the company and researchers.
\end{itemize}

\subsection{Australia Branch}
Australia branch is led by the Chief Architect Eng. Sagara.Although Research in Australia branch is not as prominent factor as in USA branch, Australia branch is more directed toward software development and maintenance. 

\section{Organizational structure of vcLABs}
\begin{figure}[!hbt]
		\begin{center}
		\includegraphics [width=.4\textwidth]{vc-organization.png}
		\caption{The organizational Structure of vcLABs }.
		\label{fig:vc-organisation}
		\end{center}
\end{figure}

The organizational structure and hierarchical levels of vcLABs is shown in Fig. \ref{fig:vc-organisation}. As the company is expecting to be expanded, the structure may be subject to change in the future.

\section{SWOT Analysis on vcLABs Colombo Chapter}
\subsection{Strengths}
\begin{itemize}
\item Only AI research company in Sri Lanka
\item Access to best passed out graduates of Sri Lanka
\item Long lasting relationship with DialogTV
\item First commercially operating ad replacement system
\item Patents in ad detection algorithm to handle more than 100000 ads
\item Challenging and dynamic environment for enthusiastic personnels
\end{itemize}
\subsection{Weaknesses}
\begin{itemize}
\item Less amount of workforce
\item Less publicity
\item Comparably lower salary than existing software related companies
\end{itemize}
\subsection{Opportunities}
\begin{itemize}
\item Less amount of players in the market
\item Attractive research areas in AI
\item Broad application scope in AI
\item Higher number of interest groups including university graduates, companies who are willing to embed AI into their products, consumers who are willing to experience AI applications
\end{itemize}
\subsection{Threats}
\begin{itemize}
\item New emerging start ups
\item Existing software companies who are migrating toward AI
\item Misconceptions about AI
\item Poorly defined government regulations on computing / AI
\item Brain-drain
\end{itemize}

\section{Other Aspects}
In this this section we discuss the profitability of vcLABs, its usefulness to the country in general and some of my suggestions to improve its performance further and possible constraints to achieve such objectives.

\subsection{Profitability of vcLABs}
Since the core product is quite stable and up and running in DialogTV the company is guaranteed to earn its income from Dialog. Since the new project is also about to release, the income will increase further more. But as the company is in the process of expanding, more workforce is about to be added incurring additional expenses.

The management is currently considering migrating the company to a PLC, which will be quite a change to the company.

\subsection{Usefulness to Sri Lanka}
In the past few centuries, the world under went major revolutions namely industrial revolution, information revolution, etc. Sri Lanka wasn't in the forefront of any of those revolution. That led Sri Lanka ending up in service center, without creating any value. It was a hard blow to the economy of Sri Lanka.

Now, a major revolution is taking place in the world. That is the AI revolution. vcLABs puts Sri Lanka in the forefront of the AI wave making Sri Lanka one of the tech creators rather than leeches. Moreover vcLABs creates jobs and contribute reducing brain drain keeping our brains o ourselves. In addition vcLABs paves way to many other enterprises to our country related to AI creating competition.

\subsection{Suggestions to Improve}
I think in order to attract university graduates who look for employment, the company should do things such as workshops, awareness programs, etc. More collaborations with higher educational institutes should be made. Even school level awareness programs could help because AI is such an attractive area of interest for school children too.

By the time the company is expanded enough more attention to research can be paid. Research areas should incorporate more practical and relevant aspects to Sri Lanka such as agriculture, tourism, etc. 

\subsection{Possible Constraints to Improve}
As Sri Lankans we are most of the time attracted to service sector. People don't like to come out from their comfort zone to be vulnerable and do something new. Such attitudes will be detrimental for the company succeed as a research company. The resources in order to carry out researches are expensive. Government should grant subsidiaries for these kinds of companies so that the objectives are achieved.